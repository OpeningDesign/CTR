%\title{Overleaf Memo Template}
% Using the xc_memo package by Rob Oakes
\documentclass[a4paper,11pt]{xc_memo}
\usepackage{lmodern}
\usepackage[utf8]{inputenc}
\usepackage[T1]{fontenc}
\usepackage[english]{babel}
\usepackage{url}
\usepackage{hyperref}
\usepackage{eurosym}
\usepackage{graphicx, lipsum}
\usepackage{ae}
\usepackage{units}
\usepackage{amsmath}

%% Edit the header section here. To include your
%% own logo, upload a file via the files menu.
\memoto{}
\memofrom{Luis}
\memosubject{OXApp. Wind loads.}
\memodate{\today}

\begin{document}
\maketitle

\section{Wind loads}

The formula in determining the design wind pressure is:

\begin{equation}
  p= q \cdot G \cdot C_p - q_i(GC_{pi})
\end{equation}

\noindent Where:
\begin{description}
  \item{$G$}: gust effect factor
  \item{$C_p$}: external pressure coefficient
  \item{$(GCpi)$}: internal pressure coefficient
  \item{$q$}: velocity pressure
\end{description}

\noindent The velocity pressure $q$ must be take as:
\begin{itemize}
\item $q= q_h$ for leeward walls, side walls and roofs (evaluated at rood mean height).
\item $q= q_z$ for windward walls, evaluated at height $z$
\item $q= q_h$ for negative internal pressure, $(-GC_{pi})$ evaluation and $q_z$ for positive internal pressure evaluation $(+GC_{pi})$ of partially enclosed buildings but can be taken as $q_h$ for conservative value.
\end{itemize}

\subsection{Gust effect factor: G}
The gust effect factor, $G$, is set to 0.85 as the structure is assumed rigid (Section 26.9.1 of ASCE 7-10).

\subsection{Enclosure classification and pressure coefficients}
The structure satisfies the definition of partial enclosed building in section 26.2 of ASCE 7-10.

\subsubsection{Internal pressure coefficient}
The internal pressure coefficient, $(GC_{pi})$, shall be +0.55 and -0.55 based on Table 26.11-1 of ASCE 7-10.

\begin{figure}
  \begin{center}
  \includegraphics[width= 60mm]{internal_pressure_coefficients_for_wall_surfaces}
  \end{center}
  \caption{Internal pressure coefficient, $(GCpi)$, from table 26.11-1 of ASCE 7-10}
\end{figure}

\subsubsection{External pressure coefficient}
For enclosed and partially enclosed buildings, the external pressure coefficient, $C_p$, is calculated using the information provided in Figure 27.4-1. For a partially enclosed building with a monoslope roof.

\begin{figure}
  \begin{center}
  \includegraphics[width= 120mm]{figure_27_4_1.png}
  \end{center}
  \caption{External pressure coefficients, $(Cp)$, from table 27.4-1 of ASCE 7-10.}
\end{figure}

\begin{figure}
  \begin{center}
  \includegraphics[width= 120mm]{wall_pressure_coefficients.png}
  \end{center}
  \caption{External pressure coefficients, $(Cp)$ on each building surface.}
\end{figure}



\end{document}
